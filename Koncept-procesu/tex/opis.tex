\section{Opis procesu}

\subsection{Oczekiwany efekt}
Proces ma na celu sprzedaż co najmniej jednego biletu na wydarzenie kulturalne
użytkownikowi aplikacji mobilnej w ramach usług systemu \emph{Concerto}.

\subsection{Zarys scenariusza głównego}
Lista zawiera główne kroki przypadku użycia, wokół którego stworzymy model
procesu biznesowego. Oznaczyliśmy miejsca ewentualnych rozgałęzień do
scenariuszy pobocznych wraz z powodem ich zajścia. Opis ścieżek alternatywnych
pominęliśmy, z nastawieniem, że zajmiemy się nimi szczegółowo później.
\begin{enumerate}[noitemsep]
    \item System sprawdza dostępność biletów na wydarzenie.
    \item Aplikacja wyświetla przycisk dla akcji kupna biletów, jeżeli są one
        dostępne. (\textbf{Rozgałęzienie:} Biletów brakuje.)
    \item Użytkownik potwierdza chęć kupna biletu na wydarzenie.
    \item Użytkownik podaje liczbę biletów, które chce zamówić.
        (\textbf{Rozgałęzienie:} Dostępnych biletów jest za mało.)
    \item (Krok opcjonalny) Użytkownik wybiera miejsca siedzące, jeżeli impreza
        na to pozwala.
    \item System pokazuje potwierdzenie zamówienia, wypełnione danymi przed
        chwilą zebranymi i danymi podanymi jako domyślne w ustawieniach
        aplikacji:
        \begin{itemize}[nosep]
            \item czas i miejsce koncertu,
            \item liczbę i miejsca biletów,
            \item cena,
            \item dane osobowe kupującego i do wysyłki,
            \item metoda płatności.
        \end{itemize}
    \item Użytkownik wyraża zgodę na dokonanie zakupu. (\textbf{Rozgałęzienie:}
        Użytkownik nie zgadza się.)
    \item System dokonuje zakupu w imieniu użytkownika.
    \item System zleca wysłanie biletów, zgodnie z danymi z potwierdzenia.
    \item Koniec.
\end{enumerate}
